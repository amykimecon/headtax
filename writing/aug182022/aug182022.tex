\documentclass[pdf]{beamer}
\mode<presentation>{}

%% Preamble
\title{Asian Immigration to the US and Canada}
\subtitle{Background and Initial Thoughts}
\author{Amy Kim}

\begin{document}

\begin{frame}
    \titlepage
\end{frame}

\begin{frame}{Introduction}
    \begin{itemize}
        \item Derenoncourt et al. (2022) inspired me to think about the effect of historical policies on Asian wealth accumulation & "Model Minority Myth"
        \begin{itemize}
            \item In particular, Head Tax in Canada and effect on initial wealth of immigrants
        \end{itemize}
        \pause
        \item Broadly, also interested in differences between Canada and US
        \begin{itemize}
            \item Similar cultural environments with geographic proximity
            \item Very different histories and legislation
            \item Some distinct differences in racial and ethnic makeup -- why?
        \end{itemize}
    \end{itemize}
\end{frame}

\begin{frame}{Historical Background: Chinese Immigration}
    \begin{itemize}
        \item 1850s: Gold Rush
        \item 1860s: Transcontinental Railroad (US)
        \item 1880s: Canadian Pacific Railroad (Canada)
        \item 1910s: Labor Shortage during WWI
    \end{itemize}
\end{frame}

\begin{frame}{Historical Background: Chinese Exclusion Act in Canada}
    \begin{itemize}
        \item 1885: \$50 Head Tax on Chinese immigrants to Canada 
        \item 1900: Increase to \$100
        \item 1903: Increase to \$500 (equiv. to \$16.3k today)
        \item Exemptions for students, merchants, diplomats ...
        \item Canadian govt collected \$23 million total over this period
        \item Chinese population still grew, nearly 10x from 1881 to 1921
        \item Chinese Immigration Act of 1923 completely banned Chinese immigration until 1947
    \end{itemize}
\end{frame}

\begin{frame}{Literature}
    \begin{itemize}
        \item Much of the literature on model minority myth is based in Sociology/
        \begin{itemize}
            \item 'Model minority myth' 
            \item Chen (2015): Skill-based immigration restrictions worsened outcomes for Chinese vs. Japanese immigrants
            \item Hilger (2016): Initially, White/Asian income gap due to discrimination, upward mobility since 1920s due to increased skill compensation (not educational mobility)
            \item Chen and Xie (2020): Discrimination caused slower "occupational assimilation" for Chinese Immigrants to US during Exclusion Act (1882-1943)
        \end{itemize}
        \item Green and Green (2014): Immigration inflows into Canada in 1910s-20s had minimal impact on overal earnings distributions through general equilibrium
        \item 
        \item Chen (2015): 
    \end{itemize}
\end{frame}

\end{document}