\section{Introduction}
Who immigrates and why? While an extensive branch of economic literature studies selection into migration, pieces of this question remain unanswered. In particular, the empirical relationship between the cost of migration and the flow and composition of immigrant groups, is not fully understood. Today, immigration policy imposes a complex web of costs, whether in the form of administrative fees, transit, immigration lawyers, or time, making the effects of these costs both crucial to understand and difficult to measure empirically. 

In this paper, I use a discriminatory historical immigration policy to study the effects of explicit costs on immigration. The `Chinese Head Tax' was a lump-sum fee uniquely levied on Chinese immigrants to Canada during the late 19th and early 20th centuries. The head tax, initially a \$50 entry fee (1,500 USD in 2023), evolved to a \$500 entry fee (14,000 USD in 2023), which was prohibitively expensive for many immigrants at the time. The combination of an immigration policy that imposed an explicit, known cost (with temporal variation) on an explicit, known group and several detailed and comprehensive data sources provides the unique opportunity to measure the precise impact of migration costs on immigration. 

I begin by [summary of paper, results].

The majority of work on immigration in a historical context uses the standard Roy model to explain selection under the assumption that migration costs are consistent across immigrant groups. 

The standard Roy model \citet{roy1951} 
A large body of literature on selection into migration begins with the standard Roy model \citet{roy1951}. 



My work relates to two main bodies of research. The first looks at the effects of historical immigration policy on both the flow and the characteristics of immigrants, primarily to the US. Nearly all of the policies studied, such as the __, __, or, perhaps most relevantly, the Chinese Exclusion Act of 1882,  

While a large number of papers have studied various historical immigration polices and their effects on the immigrant pool, no one to my knowledge has looked specifically at the Chinese Head Tax in Canada. 

The majority of immigration polices that have been studied 




While some researchers have studied historical anti-Chinese immigration policy \citep{Chen2015, ChenXie2020, Postel2023}, with meaningful results, no one to my knowledge has looked specifically at the Chinese Head Tax in Canada. Analyzing the effects of this policy on Chinese immigration inflow to Canada, and the subsequent effects of this policy on the characterstics of Chinese immigrants after arrival (whether through selection into immigration or through the wealth effects of the tax) will not only shed light on the past, but will also inform policymakers today about the consequences of immigration costs (whether explicit or implicit) on immigration inflow and immigrant outcomes.

In this paper, I find evidence that the Chinese Head Tax \textbf{did} have significant effects on Chinese immigration inflows to Canada. I also analyze outcomes for Chinese immigrants to Canada, and find evidence that the head tax had a strong negative impact on home ownership compared to other immigrant groups, even after controlling for China-side immigrant supply effects by using data on US immigrants. 
I also find evidence that the head tax increased the likelihood of Chinese immigrants being literate and decreased their likelihood of working as a laborer, suggestive of a positive effect on selection into migration (i.e. more skilled immigrants became relatively more likely to immigrate as the head tax increased). I also look at earnings, and find that effects are positive when compared to all immigrants (suggestive of a selection effect) but when compared to only Japanese immigrants, effects are negative.

Section 2 of this paper provides more historical context on the Chinese Head Tax in Canada as well as anti-Chinese immigration policy in the US at the time, section 3 describes the data sources I use for my analysis, and section 4 presents some preliminary results.
