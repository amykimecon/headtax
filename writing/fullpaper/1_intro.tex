\section{Introduction}
Who immigrates and why? While an extensive branch of economic literature studies selection into migration, many pieces of this question remain unanswered. In particular, the empirical relationship between the cost of migration and the flow and composition of immigrant groups is not fully understood. Today, immigration policy imposes a complex web of costs, whether in the form of administrative fees, transit, immigration lawyers, or time, making the effects of these costs both crucial to understand and difficult to measure empirically. 

In this paper, I use a discriminatory historical immigration policy to study the effects of explicit costs on immigration. The `Chinese Head Tax' was a lump-sum fee uniquely levied on Chinese immigrants to Canada during the late 19th and early 20th centuries. The head tax, initially a \$50 entry fee (1,500 USD in 2023), eventually evolved to a \$500 entry fee (14,000 USD in 2023), which was more than x times the average Chinese immigrant's annual salary at the time. The combination of an immigration policy that imposed an explicit, known cost (with temporal variation) on an explicit, known group and several novel data sources provides the unique opportunity to measure the direct effect of migration costs on immigration. [one-sentence summary of findings]

Previous work on historical immigration to the US and Canada is consistent with the Roy model of selection, in which migrant selectivity is dependent on the returns to skill in the receiving country relative to the sending country. [insert cites \& brief description of evidence]. While we have considerable evidence on selection into immigration during the age of mass migration, the vast majority of this body of work is focused on Europe,mostly due to a lack of origin country data. I use [description of data] to paint a broad picture of selection of Chinese immigrants into immigration to Canada in particular, compared to other destinations. Additionally, [compare to Chen 2015 and any other papers related to non-european immigration at the time] 

Present-day evidence shows much more positive selection of immigrants than would be predicted by the Roy model, particularly from countries with higher inequality such as Mexico. This discrepancy can be rationalized by skill-varying migration costs that have increased over time, `pricing out' the lower-income immigrants who would stand to gain the most from migrating, and generating intermediate or positive selection \citep{chiquiarhanson2005, mckenzierapoport2010}. 
Despite the importance of migration costs in understanding selection into migration, however, empirical evidence on the effects of migration costs is difficult to obtain due to the complexity and heterogeneity underlying the cost of migration in present-day settings. I make use of explicit variation in the cost of migration in this setting, to parse out both initial selection and the change in selection due to increases in cost. [why is this useful? what do we learn?]

[also something about literature on effects of immigration policies]

I begin by showing the direct impact of the Head Tax on Chinese immigration inflow to Canada. Using Chinese Register data I find that after controlling for other push and pull factors, the \$500 Head Tax was associated with a reduction in the annual inflow of Chinese immigrants by 8,803 relative to the \$50 Head Tax. This decrease is more than 350\% of the average inflow over the span of the Head Tax, indicating that the Head Tax was effective at deterring Chinese immigration to Canada. While significantly attenuated, likely by return migration, the Census estimates qualitatively support this finding, and placebo tests using immigration from other countries reveal no significant decreases in immigration associated with the Head Tax in other countries.

I next use Chinese Register data to study how the Head Tax affected selection into migration. I find that Chinese immigrants to Canada were initially xxly selected on height, but/and that over the course of the Head Tax period, Chinese immigrants became increasingly positively selected on height. This is consistent with the Roy model of selection under the assumption xyz, and accords with empirical evidence from other studies of the effects of migration cost on selection into immigration from higher-inequality sending countries. Additional evidence using age heaping...

Finally, I use Census data to estimate the effects of the Head Tax on outcomes after arrival for Chinese immigrants relative to other immigrants using a differences-in-differences design. I find xyz. 



% I build upon an existing body of work studying the effect of changing migration costs 

% The vast majority of papers studying the effect of migration costs make use of either cross-sectional or temporal variation in \texit{proxies} for migration costs, such as migrant networks, border patrol intensity, immigration quotas, . While useful for determining that (a) these factors do in fact affect an individual's decision to migrate, and (b) 

% The standard Roy model \citet{roy1951} 
% A large body of literature on selection into migration begins with the standard Roy model \citet{roy1951}. 



% My work relates to two main bodies of research. The first looks at the effects of historical immigration policy on both the flow and the characteristics of immigrants, primarily to the US. Nearly all of the policies studied, such as the __, __, or, perhaps most relevantly, the Chinese Exclusion Act of 1882,  

% While a large number of papers have studied various historical immigration polices and their effects on the immigrant pool, no one to my knowledge has looked specifically at the Chinese Head Tax in Canada. 

% The majority of immigration polices that have been studied 




% While some researchers have studied historical anti-Chinese immigration policy \citep{Chen2015, ChenXie2020, Postel2023}, with meaningful results, no one to my knowledge has looked specifically at the Chinese Head Tax in Canada. Analyzing the effects of this policy on Chinese immigration inflow to Canada, and the subsequent effects of this policy on the characterstics of Chinese immigrants after arrival (whether through selection into immigration or through the wealth effects of the tax) will not only shed light on the past, but will also inform policymakers today about the consequences of immigration costs (whether explicit or implicit) on immigration inflow and immigrant outcomes.

% In this paper, I find evidence that the Chinese Head Tax \textbf{did} have significant effects on Chinese immigration inflows to Canada. I also analyze outcomes for Chinese immigrants to Canada, and find evidence that the head tax had a strong negative impact on home ownership compared to other immigrant groups, even after controlling for China-side immigrant supply effects by using data on US immigrants. 
% I also find evidence that the head tax increased the likelihood of Chinese immigrants being literate and decreased their likelihood of working as a laborer, suggestive of a positive effect on selection into migration (i.e. more skilled immigrants became relatively more likely to immigrate as the head tax increased). I also look at earnings, and find that effects are positive when compared to all immigrants (suggestive of a selection effect) but when compared to only Japanese immigrants, effects are negative.

Section 2 of this paper provides more historical context on the Chinese Head Tax in Canada as well as anti-Chinese immigration policy in the US at the time, section 3 describes the data sources I use for my analysis, and section 4 presents some preliminary results.
