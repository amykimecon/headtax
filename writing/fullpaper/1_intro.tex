\section{Introduction}
Immigration has always played a key role in the development (?) of the U.S. and Canada. Today, immigrants comprise x\% of the population in the U.S. and y\% in Canada, with both of these numbers on the rise. Nearly as long as there has been immigration to the U.S. and Canada, however, there have also been efforts to limit the inflow of immigrants, motivated by fears of labor market competition, limited resources, and outright racial animosity. Historically, immigration policy was often explicit in its exclusion, imposing quotas or outright bans on immigration from specific countries. More recently, governments have attempted to combat the rising demand for a limited number of immigration slots by imposing a series of complex costs on potential immigrants, ranging from administrative fees (which have nearly xed over the past y years) to skill requirements (such as language or education) for many immigration pathways to even physical barriers, in the case of the U.S.-Mexico border.

% In recent years, immigration policy in the U.S. and Canada has increasingly imposed a series of complex costs on potential immigrants, whether in the form of administrative fees, proof of funds, skill restrictions, or time. 
Despite the growing body of research on the broad reach and striking consequences of exclusionary immigration policies \citep{clemensetal2018,ChenXie2020,feigenberg2020,Abramitzkyetal2023}, the complexity of the modern immigration system has made it difficult not only to cleanly identify how immigrants respond to changes in migration costs independent of other changes in immigration policy, but even to directly measure costs at all. The same increasing complexity, however, has emphasized the importance of understanding exactly how immigrants respond to migration costs: who is excluded when costs increase?

% Exclusionary immigration policy in the U.S. and Canada has taken many different forms over the past several centuries, from the explicit exclusion of ethnic groups in the late 19th and early 20th century (cite), to the implicit restrictions imposed by the high cost of immigration today.\footnote{avg cost to immigrate to canada today} 

% While historical immigration policy was often explicit in its exclusion, taking the form of country-level quotas or even outright bans, modern day governments have adopted a more subtle method of balancing supply and demand: increasing the cost. 

% Immigration has historically been considered a pathway to opportunity for all, particularly in North America. Over the past several centuries, immigrants have arrived in the U.S. and Canada seeking shelter from war, refuge from persecution, or a better future for themselves and their children. Such a rosy picture of immigration as a chance at a better life, however, masks a much harsher reality -- that chance is not and has never been equally available to all. U.S. and Canadian immigration policy has been designed to keep out broad swathes of the world, whether historically, in the form of explicit country-level bans and quotas, or implicitly in today's world through high costs, and the world's non-White and poor have largely borne the brunt of this exclusion.

% Despite the growing body of research on the broad reach and striking consequences of exclusionary immigration policies \citep{clemensetal2018,ChenXie2020,feigenberg2020,Abramitzkyetal2023}, the direct impact of migration costs on immigration has not been well-documented. This is largely due to the increasing complexity of costs, imposed in forms ranging from administrative fees and transit to immigration lawyers and time, which has made it difficult either to evaluate the empirical effects of costs or even to measure costs at all. The same increasing complexity, however, has emphasized the importance of understanding exactly how immigrants respond to migration costs: who is excluded when costs increase?

% Who immigrates and why? While an extensive branch of economic literature studies selection into migration, many pieces of this question remain unanswered. In particular, the empirical relationship between the cost of migration and the flow and composition of immigrant groups is not fully understood. Today, immigration policy imposes a complex web of costs, whether in the form of administrative fees, transit, immigration lawyers, or time, making the effects of these costs both crucial to understand and difficult to measure empirically. 

In this paper, I use an exclusionary immigration policy to study the effects of migration costs on immigration. The `Chinese Head Tax' was a lump-sum fee uniquely levied on Chinese immigrants to Canada during the late 19th and early 20th centuries. The head tax, initially a \$50 entry fee (1,500 USD in 2023), eventually evolved to a \$500 entry fee (14,000 USD in 2023), which was more than double the average Chinese immigrant's annual salary in Canada in 1901 \citep{census1901}. The combination of an immigration policy that imposed an explicit, time-varying cost on one specific group and rich historical data provides the unique opportunity to measure the direct effect of migration costs on immigration. By looking at Chinese immigrants specifically, I also document the effects of exclusion on an immigrant group that has been understudied despite their central role in the evolution of immigration and immigration policy in the U.S. and Canada.

I begin by showing the direct impact of the Head Tax on the magnitude of Chinese immigration to Canada. Using a comprehensive and detailed record of Chinese immigrants to Canada between 1886 and 1923, known as the Chinese Register, I find that after controlling for other factors, the Head Tax reduced Chinese immigration by nearly 9,000 immigrants (a reduction of approximately 80\%) per year at its peak. While attenuated by return migration, estimates from the Canadian census qualitatively support this finding, and placebo tests do not reveal any unexplained drops in immigration from other countries over this time period.

I next present evidence on the self-selection of the Chinese population into immigration to Canada.\footnote{Throughout this paper I refer to self-selection, the individual decision to migrate as a function of skill or human capital, as selection. Self-selection should be distinguished from imposed selection, i.e. restrictions on immigration based on skill or human capital, which is not the focus of my paper.} 
Descriptive evidence suggests that Chinese immigrants to Canada initially exhibited intermediate selection on height relative to the Chinese population but as the Head Tax increased migration costs, Chinese immigrants became increasingly positively selected on height. I confirm these descriptive findings using a difference-in-differences design comparing the characteristics of Chinese immigrants arriving in Canada at different stages of the Head Tax to other immigrants. I find that after the implementation of the Head Tax, Chinese immigrants became less likely to work as laborers, more likely to be literate, and more likely to own houses relative to other immigrant groups.  

% Using Chinese Register data I find that after controlling for other push and pull factors, the \$500 Head Tax was associated with a reduction in the annual inflow of Chinese immigrants by 8,803 relative to the \$50 Head Tax. This decrease is more than 350\% of the average inflow over the span of the Head Tax, indicating that the Head Tax was effective at deterring Chinese immigration to Canada. While significantly attenuated, likely by return migration, the Census estimates qualitatively support this finding, and placebo tests using immigration from other countries reveal no significant decreases in immigration associated with the Head Tax in other countries.

% I find that the Head Tax reduced Chinese immigration by nearly 9,000 immigrants per year at its peak, and estimate an elasticity of immigrant inflow with respect to migration cost of -0.15. In the 1880's, I find evidence of intermediate selection of Chinese immigrants into immigration to Canada relative to the Chinese population, but as th
% [one-sentence summary of findings]

Previous work on historical immigration to the U.S. and Canada is consistent with the Roy-Borjas model of selection, in which migrant selectivity is dependent on the returns to skill in the receiving country relative to the sending country \citep{roy1951,borjas1987, Abramitzkyetal2013, abramitzkyboustan2017, connor2019}. While there is considerable evidence on selection into immigration during the age of mass migration, the vast majority of this work is focused on Europe, mostly due to a lack of origin country data in non-European countries. The limited body of work centered on Chinese immigrants in the late 19th century in the context of the Chinese Exclusion Act in the U.S. is primarily based on data from the decennial U.S. Census, leading to a lack of coverage of intercensal years or accounting for return migration \citep{Chen2015,ChenXie2020}. 
I contribute to the literature on historical immigration by supplementing Canadian Census data with detailed migration microdata including the exact date of arrival to paint a broad picture of selection of Chinese immigrants into immigration to Canada in particular, compared to other destinations.

In contrast, present-day evidence shows more positive selection of immigrants than would be predicted by the Roy model, particularly from countries with higher inequality such as Mexico \citep{chiquiarhanson2005, mckenzierapoport2010}. This discrepancy can be rationalized by skill-varying migration costs that have increased over time, `pricing out' the lower-income immigrants who would stand to gain the most from migrating, and generating intermediate or positive selection. 
Despite the importance of migration costs in understanding selection into migration, however, empirical evidence on the effects of migration costs is difficult to obtain due to the complexity and heterogeneity underlying the cost of migration in present-day settings. The majority of the literature on migration costs uses indirect proxies for migration cost, such as liquidity constraints and income \citep{Angelucci2015, cai2020}, visa laws \citep{ortegaperi2013}, migrant networks \citep{mckenzierapoport2010}, border patrols or fences \citep{hansonspilimbergo1999, angelucci2012,feigenberg2020}. I contribute to the literature by making use of variation in the actual dollar cost of migration to parse out both initial selection and the change in selection due directly to increases in monetary cost. I also explicitly identify the response of immigration inflow to an increase in the cost of migration, effectively quantifying the consequences of the Head Tax and similar cost-based exclusionary policies.



% I build upon an existing body of work studying the effect of changing migration costs 

% The vast majority of papers studying the effect of migration costs make use of either cross-sectional or temporal variation in \texit{proxies} for migration costs, such as migrant networks, border patrol intensity, immigration quotas, . While useful for determining that (a) these factors do in fact affect an individual's decision to migrate, and (b) 

% The standard Roy model \citet{roy1951} 
% A large body of literature on selection into migration begins with the standard Roy model \citet{roy1951}. 



% My work relates to two main bodies of research. The first looks at the effects of historical immigration policy on both the flow and the characteristics of immigrants, primarily to the US. Nearly all of the policies studied, such as the __, __, or, perhaps most relevantly, the Chinese Exclusion Act of 1882,  

% While a large number of papers have studied various historical immigration polices and their effects on the immigrant pool, no one to my knowledge has looked specifically at the Chinese Head Tax in Canada. 

% The majority of immigration polices that have been studied 




% While some researchers have studied historical anti-Chinese immigration policy \citep{Chen2015, ChenXie2020, Postel2023}, with meaningful results, no one to my knowledge has looked specifically at the Chinese Head Tax in Canada. Analyzing the effects of this policy on Chinese immigration inflow to Canada, and the subsequent effects of this policy on the characterstics of Chinese immigrants after arrival (whether through selection into immigration or through the wealth effects of the tax) will not only shed light on the past, but will also inform policymakers today about the consequences of immigration costs (whether explicit or implicit) on immigration inflow and immigrant outcomes.

% In this paper, I find evidence that the Chinese Head Tax \textbf{did} have significant effects on Chinese immigration inflows to Canada. I also analyze outcomes for Chinese immigrants to Canada, and find evidence that the head tax had a strong negative impact on home ownership compared to other immigrant groups, even after controlling for China-side immigrant supply effects by using data on US immigrants. 
% I also find evidence that the head tax increased the likelihood of Chinese immigrants being literate and decreased their likelihood of working as a laborer, suggestive of a positive effect on selection into migration (i.e. more skilled immigrants became relatively more likely to immigrate as the head tax increased). I also look at earnings, and find that effects are positive when compared to all immigrants (suggestive of a selection effect) but when compared to only Japanese immigrants, effects are negative.

Section 2 of this paper provides more historical context on the Chinese Head Tax in Canada as well historical immigration to Canada more broadly, section 3 describes the data sources I use for my analysis, and section 4 presents results on the effects of the Head Tax on the inflow of Chinese immigrants. Section 5 describes the theoretical framework I use to analyze selection, and section 6 presents both descriptive and causal effects of the Head Tax on the selection of potential Chinese migrants into immigration to Canada.
