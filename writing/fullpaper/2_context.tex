\section{Background and Historical Context}
Although the history of Chinese immigration to North America reaches back to the 1700s, it was not until the Gold Rush of the 1850s that large numbers of Chinese immigrants began to settle along the West Coast. 
While initially concentrated in California, later discoveries of gold in British Columbia expanded Chinese immigration to Canada, and by 1860 there were an estimated 7,000 Chinese inhabitants of British Columbia \citep{chan2019}.
Following the Gold Rush, Chinese immigrants largely took on low-wage jobs in the manufacturing and service industries, and contributed heavily to the construction of the Canadian Pacific Railway during the early 1880s.
As the completion of the railway approached, sinophobic sentiment began to foment and in 1884 the Royal Commission of Chinese Immigration was formed to investigate the possibility of restricting Chinese immigration to Canada \citep{chan2016}.

\subsection{Chinese Head Tax in Canada}
With the passage of the Chinese Immigration Act on July 20th, 1885, came a \$50 `Head Tax' -- a per-person flat entry fee for all immigrants from China. While the Royal Commission of Chinese Immigration had originally suggested a \$10 head tax, intended to pay for health inspectors at entry ports, it was clear that the higher \$50 fee was implemented with the goal of dissuading Chinese immigrants from settling in Canada. 
The \$50 Head Tax (approximately 1,500 USD in 2023) was roughly equivalent to the cost of the cheapest one-way ticket from East Asia to the West coast, effectively doubling the cost of immigration from China to Canada \citep{boatcost}. As Chinese immigration to Canada continued to grow despite the tax, the government raised the tax to \$100 on January 1st, 1901 (approximately 3,000 USD in 2023) and again to \$500 on January 1st, 1904 (approximately 14,000 USD in 2023).\footnote{The \$100 and \$500 amendments to the Chinese Head Tax were passed in July of 1900 and 1903 respectively, but only went into effect on January 1st of the following year.} In comparison, the average adult Chinese immigrant man's annual earnings, as recorded in the 1901 Canadian Census, totalled only \$235. The tax remained in place until 1923, when the Chinese Immigration Act of 1923 banned Chinese immigration to Canada altogether until its repeal in 1947.

While exceptions to the tax were made for diplomats, merchants, students, and others, the vast majority of Chinese immigrants were forced to pay the tax. Failure to do so would result in being deported or imprisoned, and as a result many immigrants had to either borrow money or enter indentured servitude contracts to cover the cost of the tax. Appendix Figure \ref{fig:taxpaid}, which plots the average non-zero tax paid by Chinese immigrants in each year, suggests nearly perfect adherence to the Chinese Head Tax by port officials.\footnote{Note that only 8.7\% of pre-1923 arrivals recorded in the Chinese Register paid \$0 in taxes. Of these non-taxpaying individuals, approximately 89\% were merchants or the family of a merchant, and an additional 4\% were students or working in other professions exempt from the tax.} 
While not all Chinese immigrants who arrived in Canada prior to 1885 were required to pay the tax, Appendix Figure \ref{fig:taxpaid} indicates that some pre-1885 arrivals did pay, likely in order to be able to freely exit and re-enter Canada. While these pre-1885 arrivals largely paid the \$50 head tax (suggesting re-entry to Canada between 1885 and 1900), small spikes prior to 1885 indicate that some were required to pay more than \$50 (approximately 6\% of all immigrants in the Register data who arrived prior to 1885), suggesting re-entry in later years.\footnote{While the Chinese Immigration Act initially permitted Chinese immigrants to leave and re-enter the country without repaying the Head Tax, an 1892 amendment required repayment of the tax if a Chinese immigrant left Canada for more than twelve months.} 

The Chinese Immigration Act and its subsequent amendments also included a number of other restrictions on Chinese immigration, such as a capacity limit for incoming ships carrying Chinese passengers and the complete prohibition of ``any person of Chinese origin who is a pauper or likely to become a public charge" from entry into Canada.\footnote{Statutes of Canada. An Act Respecting and Restricting Chinese Immigration, 1900. Ottawa: SC 63-64 Victoria, Chapter 32.}
Even beyond immigration restrictions, discrimination against Chinese immigrants during this time period was widespread, including federal disenfranchisement by the Electoral Franchise Act of 1885 and the Dominion Elections Act of 1898, and riots in protest of continued Chinese immigration \citep{pier21context}.

\subsection{Canadian Immigration from Other Countries}
In this paper, I compare Chinese immigrants with two different immigrant groups: all non-Chinese immigrants and Japanese immigrants. In this section, I briefly summarize the broader historical context around immigration to Canada -- I present summary statistics comparing these groups in section 3.3 and discuss in further detail the specific assumptions required to compare these groups under my empirical design in section 6.2.

% Using the entire population of immigrants results in a large sample size and little concern of spillover effects of the Head tax, and using Japanese immigrants 

Overall, immigration to Canada stayed relatively flat between 1880 and 1900, and then began to rapidly increase at the turn of the century until an abrupt drop in immigration in 1914 due to World War I, after which immigration stayed relatively low. While the Chinese Immigration Act was the first ever restriction on immigration to Canada for a specific ethnic group, as immigration to Canada accelerated at the turn of the century, other restrictive and xenophobic immigration policies began to emerge. The Continuous Journey Regulation of 1908, for instance, prohibited the entry of immigrants who did not travel directly to Canada from their country of origin, effectively cutting off immigration from India and other countries without direct passage routes to Canada. 
In the same year, the Canadian and Japanese governments negotiated a `Gentlemen's agreement', which capped the number of Japanese immigrants to Canada at 400 annually \citep{pier21canimm}.

In 1919, the federal government passed an amendment to the Immigration Act which permitted banning immigrant groups unsuited to Canada due to their “peculiar customs, habits, modes of life and methods of holding property and because of their probable inability to become readily assimilated.”\footnote{Statutes of Canada. An Act to Amend the Immigration Act, 1919. Ottawa: SC 9-10 George V, Chapter 25.} 
Essentially, while immigrant groups other than Chinese immigrants were largely unrestricted in immigrating to Canada for the majority of the span of the Chinese Head Tax, by 1908 this was no longer the case for ethnic minorities, and by 1914, all immigrants experienced restricted mobility due to World War I and subsequent legislation. As a result, I focus on immigration prior to 1908 when comparing Chinese immigrants to Japanese immigrants, and I focus on immigration prior to 1914 when comparing Chinese immigrants to all immigrants.


% \subsection{Chinese Immigration to the US}
% Chinese immigration to the US evolved in a similar way as Canada throughout the 19th century, beginning with the Gold Rush, growing through the construction of the Transcontinental Railroad, and culminating in the Chinese Exclusion Act of 1882.
% The legislation in the US, however, took a more severe approach than the Canadian counterpart, completely banning the immigration of Chinese laborers, a ban which remained in place until 1943. 
% While there are some accounts of immigrants from China landing in Canada and crossing the border by land into the US, estimates suggest that the Exclusion Act did heavily limit the growth of the Chinese population in the US, particularly for low-wage workers.