\section{Background and Historical Context}
Although the history of Chinese immigration to North America reaches back to the 1700s, it was not until the Gold Rush of the 1850s that large numbers of Chinese immigrants began to settle along the West Coast. 
While initially concentrated in California, later discoveries of gold in British Columbia expanded Chinese immigration to Canada, and by 1860 there were an estimated 7,000 Chinese inhabitants of British Columbia \citep{chan2019}.
Following the Gold Rush, Chinese immigrants largely took on low-wage jobs in the manufacturing and service industries, and in 1880-1885 contributed heavily to the construction of the Canadian Pacific Railway.
As the completion of the railway approached, however, sinophobic sentiment began to foment, and in 1884 a commission was formed to investigate the possibility of restricting Chinese immigration to Canada \citep{chan2016}.

\subsection{Chinese Head Tax in Canada}
With the passage of the Chinese Immigration Act of 1885 came a \$50 `Head Tax' -- a per-person entry fee for all immigrants from China. While the commission appointed to investigate Chinese immigration had originally suggested a \$10 head tax, intended to pay for a health inspector at entry ports,
it was clear that the \$50 fee was implemented with the goal of dissuading Chinese immigrants from settling in Canada. 50 CAD in 1885 would be worth approximately 1,500 USD in 2023, and at the time was nearly a fifth of the average Chinese immigrant's yearly salary.
As Chinese immigration to Canada continued to grow despite the tax, the government raised the tax to \$100 in 1900 (approximately 3,000 USD in 2023) and to \$500 in 1903 (approximately 14,000 USD in 2023). 

The Chinese Immigration Act also limited the capacity of incoming ships carrying Chinese passengers (with no changes to capacity for incoming ships carrying European passengers) and while exceptions to the tax were made for diplomats, merchants, students, and others, the vast majority of Chinese immigrants were forced to pay the tax. 
Failure to do so would result in being sent back to China, and as a result many immigrants had to either borrow money or enter indentured servitude contracts to cover the cost of the tax. 
The tax remained in place until 1923, when the Chinese Immigration Act of 1923 banned Chinese immigration to Canada altogether until its repeal in 1947.

[to edit] I begin by documenting the average non-zero tax paid by Chinese immigrants in each year, as recorded in the Chinese Register, in Figure \ref{fig:taxpaid}.\footnote{Note that only 8.7\% of pre-1923 arrivals recorded in the Chinese Register paid \$0 in taxes. 
Of these non-taxpaying individuals, approximately 89\% were merchants or the family of a merchant, and an additional 4\% were students or working in other professions exempt from the tax.} The figure suggests nearly perfect adherence to the Chinese Head Tax by port officials, with immigrants entering between 1885 and 1900 paying \$50, immigrants entering between 1900 and 1903 paying \$100, and immigrants entering between 1903 and 1923 paying \$500. 
While not all Chinese immigrants who arrived in Canada prior to 1885 were required to pay the tax, Figure \ref{fig:taxpaid} indicates that some pre-1885 arrivals did pay, likely in order to be able to freely exit and re-enter Canada. While these pre-1885 arrivals largely paid the \$50 head tax (suggesting re-entry to Canada between 1885 and 1900), small spikes prior to 1885 indicate that some were required to pay more than \$50 (approximately 6\% of all immigrants in the Register data who arrived prior to 1885), suggesting re-entry in later years. 

While the Register does not account for immigrants who entered without registration, there is reason to believe that this was a rare occurence. At the time, the only way for Chinese immigrants to arrive in Canada was by ship, since the US at the time had even stricter restrictions on Chinese immigration.\footnote{There are even some accounts of Chinese immigrants arriving by ship in Canada and attempting to cross the border \textit{into the US }by land, to circumvent the Chinese Exclusion Act in the US.} Since shipping ports were much better surveilled than land borders, it is unlikely that many would have been able to enter without registering. 

\subsection{Chinese Immigration to the US}
Chinese immigration to the US evolved in a similar way as Canada throughout the 19th century, beginning with the Gold Rush, growing through the construction of the Transcontinental Railroad, and culminating in the Chinese Exclusion Act of 1882.
The legislation in the US, however, took a more severe approach than the Canadian counterpart, completely banning the immigration of Chinese laborers, a ban which remained in place until 1943. 
While there are some accounts of immigrants from China landing in Canada and crossing the border by land into the US, estimates suggest that the Exclusion Act did heavily limit the growth of the Chinese population in the US, particularly for low-wage workers.