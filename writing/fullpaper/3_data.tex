\section{Data}
\subsection{Migration Data}
\subsubsection{Chinese Register}
My first and primary source of data is the Register of Chinese Immigrants to Canada, which is a record maintained from 1885 to 1949 of all Chinese immigrants to Canada \citep{chineseregister}. Upon entry into Canada, immigration officials were required to record in this ``Chinese Register'' the full name, age, sex, occupation, and height of each Chinese immigrant, in addition to the amount of tax paid by the immigrant and the port, date, and method of arrival.
In addition to serving as a full count of immigrants of Chinese origin who \textit{entered} Canada during the years affected by the head tax,
the Register also records any Chinese immigrants who had already entered Canada prior to the implementation of the Chinese Immigration Act in 1885 (and therefore were not subject to the tax),
but chose to pay the tax and register with immigration authorities so that they could exit and re-enter the country without paying additional fees.

Although this dataset allows me to precisely observing the timing of entry into Canada, it does not record the outflow of Chinese immigrants, and so it does not accurately capture the stock of Chinese immigrants in Canada in any given year, and does not account for Chinese immigrants who entered without registering.\footnote{There is reason to believe that very few Chinese immigrants were able to enter the country illegally. Passenger ships between Hong Kong and Vancouver or Victoria, the primary mode of transportation for Chinese immigrants to Canada, were run by a Canadian company on set schedules, and meticulous accounting of cargo would have made illegal entry by sea nearly impossible. 
Entry by land would only have been possible by way of the United States, where there were even stricter restrictions on Chinese immigration. In fact, there are even accounts of Chinese immigrants arriving by ship in Canada and attempting to cross the border \textit{into} the US by land, to circumvent the US Chinese Exclusion Act.}

\subsubsection{Hong Kong Harbourmaster Reports}
I digitize annual reports by the Harbourmaster of Hong Kong between 1870 and 1930, a novel data set with rich information on migration to and from Hong Kong. In particular, these reports record the total number of passengers departing from Hong Kong on ``passenger ships cleared by the emigration officer'', separately by destination port. By summing total emigrants to Vancouver and Victoria, I construct an annual measure of emigration from Hong Kong to Canada.\footnote{98.2\% of Chinese immigrants in the Chinese Register are recorded as having arrived in either Vancouver or Victoria.} 
Since passenger ship lines from China to North America almost exclusively departed from Hong Kong, data from Hong Kong ports should accurately capture direct emigration to Canada. Although immigrants who departed from other ports in China to intermediate countries such as Japan or Indonesia en route to Canada would not be counted in these data, this was relatively rare.\footnote{Between 1886 and 1896, when the Harbourmaster Reports include lists of the names of all emigrant ships departing Hong Kong, over 90\% of Chinese immigrants recorded in the Chinese Register listed as their method of conveyance the name of a Hong Kong passenger ship bound for either Vancouver or Victoria. This figure likely understates the true percentage of Chinese immigrants that arrived directly from Hong Kong, since for a number of Chinese immigrants, the method of conveyance was either misspelled beyond recognition or listed simply as `ship'.} 

Appendix Figure \ref{fig:immandem} compares the number of Chinese emigrants from Hong Kong to Canada as reported in the Harbourmaster reports to the number of Chinese immigrants to Canada as reported in the Chinese Register from 1886 to 1923. In the first several years of the Head Tax, reported emigration aligns almost perfectly with reported immigration, but beginning in 1893, reported emigration consistently exceeds reported immigration, in some years by more than 5,000 migrants. 
This discrepancy can be explained in three ways. 
The first is that Chinese immigrants to Canada were permitted to leave and re-enter the country within a year without re-paying the Head Tax (or being re-recorded in the Register) upon re-entry. As a result, a returning Chinese immigrant who had already paid the Head Tax would be counted in the emigration data as a passenger on an emigrant ship to Canada, but would not be counted in the Register data as a new immigrant. 
The second is that beginning in 1887, migrants who were passing through Canada en route to another country were exempted from paying the tax, and also would not have been recorded in the Register.\footnote{Although it is possible that some Chinese emigrants were turned away because they could not afford to pay the Head Tax, historian Arlene Chan has told me that Chinese immigrants were generally aware of the tax prior to coming to Canada, indicating that deportation is unlikely to explain much of the discrepancy between emigration and immigration.}
Finally, it is unclear whether the emigration data include only Chinese passengers. If they include non-Chinese passengers, such as British citizens travelling between parts of the Commonwealth, then the discrepancy between emigration and immigration may be in part due to non-Chinese passengers, who would not be recorded as immigrants in the Chinese Register.

\subsection{Census Data}
I supplement the above annual migration data with microdata from the 1881-1921 decennial Canadian censuses.\footnote{Data are compiled from \citet{census1881}, \citet{census1891}, \citet{census1901}, \citet{census1911}, \citet{census1921}, for the 1881, 1891, 1901, 1911, and 1921 censuses respectively.} While 100\% of the 1881 census has been digitized, only a random sample of the other censuses, ranging from 4-5\% coverage of the population, has been digitized, resulting in a relatively small sample size. Additionally, since the variable Year of Immigration, indicating a respondent's year of arrival in Canada conditional on being an immigrant, is only available beginning in 1901, I cannot use prior censuses for analysis of immigrant outcomes by year of arrival. I therefore use the 1881 and 1891 censuses only to measure the population stock of immigrant groups.

Nevertheless, the extensive set of variables included in the census create a rich cross-sectional view of the country at different points in time, acting as a complement to the data on inflows from the Chinese Register. In addition to variables not available in migration data, such as marital status, literacy, earnings, and home ownership, unlike the migration data the census data also include non-Chinese immigrants. Data on other immigrants allow me to directly compare the characteristics of Chinese immigrants to other immigrant groups who were not subject to the Chinese Head Tax, but otherwise would likely have experienced similar immigration pushes and pulls.

Finally, I also use US census data in my analysis. I use full-count census data from the 1900-1920 US censuses,\footnote{All data is obtained from \citet{ipums}.} which, much like the Canadian census data, serves as a cross-sectional view of the Chinese immigrant population in the US at different points in time. 

\subsection{Descriptive Statistics [to redo]}

\begin{table}[!h]
    \centering 
    \renewcommand{\arraystretch}{1.5}
    \resizebox{\textwidth}{!}{
    \begin{threeparttable}
        \caption{Summary statistics for pooled Canadian census data (1901-1921), pooled US census data (1900-1920), and the Chinese Register, grouped separately by the entire immigrant population, the Chinese immigrant population, and the Japanese immigrant population (as a comparison group).}
        \label{tab:summstats}
        % latex table generated in R 4.0.3 by xtable 1.8-4 package
% Mon Oct 10 23:48:19 2022
\begin{tabular}{lccc}
  \hline
 & CA Census (1901-1921) & Chinese Registry (1885-1924) & US Census (1900-1920) \\ 
  \hline
\% Male & 0.98 & 0.98 & 0.95 \\ 
   & (0.16) & (0.15) & (0.21) \\ 
  \% Married & 0.55 & - & 0.44 \\ 
    & (0.5) & - & (0.5) \\ 
  Avg. Age at Imm & 23.20 & 25.95 & 20.92 \\ 
     & (8.99) & (21.86) & (9.78) \\ 
  \% Can Read & 0.68 & - & 0.77 \\ 
        & (0.47) & - & (0.42) \\ 
  \% Labourers & 0.20 & 0.62 & 0.32 \\ 
           & (0.4) & (0.48) & (0.47) \\ 
   \hline
Obs &   4230 &  90946 & 186523 \\ 
   \hline
\end{tabular}

        \begin{tablenotes}
            \item *Calculated as a percentage of adults. 
            
            \item \textbf{Notes:} Column (1) pools Canadian census data (as described in section 3.1.2) from 1901, 1911, 1921, calculating means and standard errors of each variable for all individuals born outside of Canada, weighted based on the coverage of that year's census (5\% for 1901-1911, and 4\% for 1921). The small number of observations for Canada relative to the US is due to this sampling structure.
            Column (3) repeats this exercise for individuals born in China, and Column (6) repeats this exercise for individuals born in Japan. Column (2) pools American census data (as described in section 3.1.3) from 1900, 1910, 1920 to calculate means and standard errors of each variable for all individuals born outside of the US -- note that since these censuses all have 100\% coverage, there is no population weighting.
            Column (4) repeats this exercise for individuals born in China, and Column (7) repeats this exercise for individuals born in Japan. Finally, column (5) uses the entire Chinese Register (as described in section 3.1.1) to calculate means and standard errors of each variable for the population of Chinese immigrants to Canada recorded in the Chinese Register.
        \end{tablenotes}
    \end{threeparttable}
    }
\end{table}

Table \ref{tab:summstats} compares summary statistics across the different data sources, for the entire population, the entire immigrant population, and the Chinese immigrant population. Columns (1) and (2) compare the entire population of Canada, as summarized in the Canadian censuses from 1901-1921, to the entire population of the US, as summarized in the US censuses from 1900-1920. There are very small differences in characteristics between the two groups, suggesting that the general population of the two countries was very similar over this time period. 

Columns (3) and (4) compare the immigrant populations of the two countries. While also largely similar in terms of the gender distribution (slightly more male than the overall population), marital status (slightly more likely to be married than the overall population), there are several differences. The first and most significant is the difference in year of immigration -- immigrants to Canada immigrated nine years later on average. Additionally, while immigrants in both countries are older than the overall population, this is more pronounced in the US, and immigrants to the US are slightly less likely to be literate in English. 

Columns (5) and (6) compare the Chinese immigrant population of the US and Canada, focusing on those who immigrated after 1885. Observe that many of the differences between these two columns could be explained by the enforcement of the Chinese Exclusion Act in the US starting in 1882: as the inflow of lower-skilled Chinese laborers stops, the fraction of laborers in the Chinese immigrant population drops, the likelihood of literacy increases, the average age increases, and the population is slightly less male. Note, however, that this explanation does not account for the differences in likelihood of being married, and furthermore, the presence of some laborers in the post-1885 US sample suggests imperfect compliance to the Chinese Exclusion Act. 

Finally, column (7) summarizes available data from the Chinese Register. The results show that while the immigrants recorded in the Chinese Register are similarly male and arrive at similar times as the Chinese immigrants recorded in the Canadian census, they are much younger and also possibly more likely to be laborers.\footnote{For census data, I directly impute laborer status as a function of occupational group, which is encoded differently across censuses, and is therefore not directly comparable to the laborer variable in the Chinese Register, where ``Laborer'' was a category of occupation that was directly recorded. Additionally, this variable in the Chinese Register was recorded at the time of entry, and does not reflect the occupation an immigrant may have had after arrival in Canada.}
Note that the key difference between the Canadian census and the Register is that while the Register counts each (legal) entry into the country, the census is cross-sectional and therefore does not count immigrants who both arrive and subsequently leave the country between censuses. Indeed, a simple accounting exercise shows that between 1911 and 1921, the population of Chinese immigrants represented in the census grew by nearly 10,000, while in the same time period, the Chinese Register recorded the arrival of over 29,000 Chinese immigrants, implying a net outflow of 19,000 Chinese immigrants. Repeating this same exercise implies similarly large outflows in other decades in this period. 
