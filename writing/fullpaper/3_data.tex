\section{Data}
\subsection{Description of Data Sources [to be edited]}

\subsubsection{Chinese Register}
The first and primary source of data that I use is the Register of Chinese Immigrants to Canada\footnote{Data is obtained from \citet{chineseregister}.}, 
which is a record maintained from 1885 to 1949 of all Chinese immigrants to Canada.
The Register includes the full name, amount of tax paid, age and sex, occupation, place of origin within China, port and date of arrival, among other variables. 
In addition to serving as a full count of immigrants of Chinese origin who \textit{entered} Canada during the years affected by the head tax,
the Register also records any Chinese immigrants who had already entered Canada prior to the implementation of the Chinese Immigration Act in 1885 (and therefore were not subject to the tax),
but chose to pay the tax and register with immigration authorities so that they could exit and re-enter the country without paying additional fees.
Although this dataset allows me to precisely observing the timing of entry into Canada, it does not record the outflow of Chinese immigrants, and so it does not accurately capture the stock of Chinese immigrants in Canada in any given year.
Additionally, this dataset is likely to miscount Chinese immigrants who entered illegally or under a false name.

\subsubsection{Canadian Census}
Microdata from historical Canadian censuses are currently available through 1921.\footnote{Data are compiled from \citet{census1901}, \citet{census1911}, \citet{census1921}, for the 1901, 1911, and 1921 censuses respectively. Note that while data from the 1852, 1871, 1881, and 1891 censuses are also available, none contain the variable Year of Immigration, making it difficult to measure the impact of the head tax (which went into effect and changed dollar amounts between census years). As a result, these earlier censues are excluded from this analysis.} 
Unfortunately, only 4-5\% of these censuses have been digitized (with random sampling). While this data source provides a rich set of variables, the low sampling rate results in a very small sample size for Chinese immigrants, who were already a relatively small percentage of the Canadian population at the time. 
Since the census only took place every ten years, this data source provides a cross-sectional view of the country at different points in time, acting as a complement to the data on inflows from the Chinese Register.
Section 3.2 compares these two datasets in further detail.

\subsubsection{US Census}
Finally, I also use US census data in my analysis. I use full-count census data from the 1900-1920 US censuses,\footnote{All data is obtained from \citet{ipums}.} which, much like the Canadian census data, serves as a cross-sectional view of the Chinese immigrant population in the US at different points in time.

\subsection{Comparing Data Sources}

\begin{table}[!h]
    \centering 
    \renewcommand{\arraystretch}{1.5}
    \resizebox{\textwidth}{!}{
    \begin{threeparttable}
        \caption{Summary statistics for pooled Canadian census data (1901-1921), pooled US census data (1900-1920), and the Chinese Register, grouped separately by the entire immigrant population, the Chinese immigrant population, and the Japanese immigrant population (as a comparison group).}
        \label{tab:summstats}
        % latex table generated in R 4.0.3 by xtable 1.8-4 package
% Mon Oct 10 23:48:19 2022
\begin{tabular}{lccc}
  \hline
 & CA Census (1901-1921) & Chinese Registry (1885-1924) & US Census (1900-1920) \\ 
  \hline
\% Male & 0.98 & 0.98 & 0.95 \\ 
   & (0.16) & (0.15) & (0.21) \\ 
  \% Married & 0.55 & - & 0.44 \\ 
    & (0.5) & - & (0.5) \\ 
  Avg. Age at Imm & 23.20 & 25.95 & 20.92 \\ 
     & (8.99) & (21.86) & (9.78) \\ 
  \% Can Read & 0.68 & - & 0.77 \\ 
        & (0.47) & - & (0.42) \\ 
  \% Labourers & 0.20 & 0.62 & 0.32 \\ 
           & (0.4) & (0.48) & (0.47) \\ 
   \hline
Obs &   4230 &  90946 & 186523 \\ 
   \hline
\end{tabular}

        \begin{tablenotes}
            \item *Calculated as a percentage of adults. 
            
            \item \textbf{Notes:} Column (1) pools Canadian census data (as described in section 3.1.2) from 1901, 1911, 1921, calculating means and standard errors of each variable for all individuals born outside of Canada, weighted based on the coverage of that year's census (5\% for 1901-1911, and 4\% for 1921). The small number of observations for Canada relative to the US is due to this sampling structure.
            Column (3) repeats this exercise for individuals born in China, and Column (6) repeats this exercise for individuals born in Japan. Column (2) pools American census data (as described in section 3.1.3) from 1900, 1910, 1920 to calculate means and standard errors of each variable for all individuals born outside of the US -- note that since these censuses all have 100\% coverage, there is no population weighting.
            Column (4) repeats this exercise for individuals born in China, and Column (7) repeats this exercise for individuals born in Japan. Finally, column (5) uses the entire Chinese Register (as described in section 3.1.1) to calculate means and standard errors of each variable for the population of Chinese immigrants to Canada recorded in the Chinese Register.
        \end{tablenotes}
    \end{threeparttable}
    }
\end{table}

Table \ref{tab:summstats} compares summary statistics across the different data sources, for the entire population, the entire immigrant population, and the Chinese immigrant population. Columns (1) and (2) compare the entire population of Canada, as summarized in the Canadian censuses from 1901-1921, to the entire population of the US, as summarized in the US censuses from 1900-1920. There are very small differences in characteristics between the two groups, suggesting that the general population of the two countries was very similar over this time period. 

Columns (3) and (4) compare the immigrant populations of the two countries. While also largely similar in terms of the gender distribution (slightly more male than the overall population), marital status (slightly more likely to be married than the overall population), there are several differences. The first and most significant is the difference in year of immigration -- immigrants to Canada immigrated nine years later on average. Additionally, while immigrants in both countries are older than the overall population, this is more pronounced in the US, and immigrants to the US are slightly less likely to be literate in English. 

Columns (5) and (6) compare the Chinese immigrant population of the US and Canada, focusing on those who immigrated after 1885. Observe that many of the differences between these two columns could be explained by the enforcement of the Chinese Exclusion Act in the US starting in 1882: as the inflow of lower-skilled Chinese laborers stops, the fraction of laborers in the Chinese immigrant population drops, the likelihood of literacy increases, the average age increases, and the population is slightly less male. Note, however, that this explanation does not account for the differences in likelihood of being married, and furthermore, the presence of some laborers in the post-1885 US sample suggests imperfect compliance to the Chinese Exclusion Act. 

Finally, column (7) summarizes available data from the Chinese Register. The results show that while the immigrants recorded in the Chinese Register are similarly male and arrive at similar times as the Chinese immigrants recorded in the Canadian census, they are much younger and also possibly more likely to be laborers.\footnote{For census data, I directly impute laborer status as a function of occupational group, which is encoded differently across censuses, and is therefore not directly comparable to the laborer variable in the Chinese Register, where ``Laborer'' was a category of occupation that was directly recorded. Additionally, this variable in the Chinese Register was recorded at the time of entry, and does not reflect the occupation an immigrant may have had after arrival in Canada.}
Note that the key difference between the Canadian census and the Register is that while the Register counts each (legal) entry into the country, the census is cross-sectional and therefore does not count immigrants who both arrive and subsequently leave the country between censuses. Indeed, a simple accounting exercise shows that between 1911 and 1921, the population of Chinese immigrants represented in the census grew by nearly 10,000, while in the same time period, the Chinese Register recorded the arrival of over 29,000 Chinese immigrants, implying a net outflow of 19,000 Chinese immigrants. Repeating this same exercise implies similarly large outflows in other decades in this period. 
