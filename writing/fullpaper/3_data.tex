\section{Data}
\subsection{Migration Data}
\subsubsection{Chinese Register}
My first and primary source of data is the Register of Chinese Immigrants to Canada, which is a record maintained from 1885 to 1949 of all Chinese immigrants to Canada \citep{chineseregister}. Upon entry into Canada, immigration officials were required to record in this ``Chinese Register'' the full name, age, sex, occupation, and height of each Chinese immigrant, as well as the amount of tax paid by the immigrant and the port, date, and method of arrival.
In addition to serving as a full count of immigrants of Chinese origin who entered Canada during the years affected by the head tax,
the Register also records any Chinese immigrants who had already entered Canada prior to the implementation of the Chinese Immigration Act in 1885 (and therefore were not subject to the tax),
but chose to pay the tax and register with immigration authorities so that they could exit and re-enter the country without paying additional fees.

Although this dataset allows me to precisely observing the timing of entry into Canada, it does not record the outflow of Chinese immigrants, and so it does not accurately capture the stock of Chinese immigrants in Canada in any given year.\footnote{While the Register also excludes Chinese immigrants who entered illegally, i.e. without registering, there is reason to believe that very few Chinese immigrants were able to do so. Passenger ships between Hong Kong and Vancouver or Victoria, the primary mode of transportation for Chinese immigrants to Canada, were run by a Canadian company on set schedules, and meticulous accounting of cargo would have made illegal entry by sea nearly impossible. 
Entry by land would only have been possible by way of the United States, where there were even stricter restrictions on Chinese immigration. In fact, there are even accounts of Chinese immigrants arriving by ship in Canada and attempting to cross the border \textit{into} the US by land, to circumvent the US Chinese Exclusion Act.}

\subsubsection{Hong Kong Harbourmaster Reports}
I digitize annual reports by the Harbourmaster of Hong Kong between 1870 and 1930, a novel data set with rich information on migration to and from Hong Kong \citep{hkharbourmaster}. In particular, these reports record the total number of passengers departing from Hong Kong on ``passenger ships cleared by the emigration officer'', separately by destination port. By summing total emigrants to Vancouver and Victoria, I construct an annual measure of emigration from Hong Kong to Canada.\footnote{98.2\% of Chinese immigrants in the Chinese Register are recorded as having arrived in either Vancouver or Victoria.} 
Since passenger ship lines from China to North America almost exclusively departed from Hong Kong, data from Hong Kong ports should accurately capture direct emigration to Canada. Although immigrants who departed from other ports in China to intermediate countries such as Japan or Indonesia en route to Canada would not be counted in these data, this was relatively rare.\footnote{Between 1886 and 1896, when the Harbourmaster reports include lists of the names of all emigrant ships departing Hong Kong, over 90\% of Chinese immigrants recorded in the Chinese Register listed as their method of conveyance the name of a Hong Kong passenger ship bound for either Vancouver or Victoria. 
Estimating the percentage of immigrants who arrived directly from Hong Kong via ship using the method of conveyance listed in the Register likely understates the true percentage of Chinese immigrants that arrived directly from Hong Kong, since for some Chinese immigrants, the method of conveyance was either misspelled beyond recognition or listed simply as `ship'.} 
Appendix B compares migration estimates from the Harbourmaster report data and the Chinese Register data. I find in Appendix Figure \ref{fig:immandem} that while the magnitudes implied by the two data sources diverge in later years, the overall peaks and troughs of migration are remarkably similar.

\subsection{Census Data}
I supplement the above annual migration data with microdata from the 1881-1921 decennial Canadian censuses.\footnote{Data are compiled from \citet{census1881}, \citet{census1891}, \citet{census1901}, \citet{census1911}, \citet{census1921}, for the 1881, 1891, 1901, 1911, and 1921 censuses respectively.} While 100\% of the 1881 census has been digitized, only a random sample of the other censuses, ranging from 4-5\% coverage of the population, has been digitized, resulting in a relatively small sample size. Additionally, since the variable Year of Immigration, indicating a respondent's year of arrival in Canada conditional on being an immigrant, is only available beginning in 1901, I cannot use prior censuses for analysis of immigrant outcomes by year of arrival and thus I use the 1881 and 1891 censuses only to measure the population stock of immigrant groups. 
Another major obstacle in using the census data is attrition due to return migration. If return migrants are differentially selected from the immigrant population, then the characteristics of immigrants who arrived in Canada in 1880 as measured in the 1901 census would not be representative of the full population of immigrants who arrived in Canada in 1880. 

Nevertheless, the extensive set of variables included in the census create a rich cross-sectional view of the country at different points in time, acting as a complement to the detailed data on inflows from the Chinese Register, which does not suffer bias from return migration. In addition to having variables not available in migration data, such as marital status, literacy, earnings, and home ownership, the census data also include non-Chinese immigrants, unlike the migration data. Data on other immigrants allow me to directly compare the characteristics of Chinese immigrants to other immigrant groups who were not subject to the Chinese Head Tax, but otherwise would likely have experienced similar immigration pushes and pulls. 
Comparing similar immigrant groups also ameliorates the selection problem caused by return migration -- if we think for instance that Japanese and Chinese immigrants would have been similarly selected into return migration in the absence of the Head Tax, then a differences-in-differences empirical strategy would account for return migration, although the estimated effect of the Head Tax on Chinese immigrant characteristics would then be a combination of the effect of the Head Tax on selection into immigration and the effect of the Head Tax on selection into return migration.

% Finally, I also use US census data in my analysis. I use full-count census data from the 1900-1920 US censuses,\footnote{All data is obtained from \citet{ipums}.} which, much like the Canadian census data, serves as a cross-sectional view of the Chinese immigrant population in the US at different points in time. 

\subsection{Summary Statistics}

\begin{table}[!h]
    \centering 
    \renewcommand{\arraystretch}{1.5}
    \resizebox{\textwidth}{!}{
    \begin{threeparttable}
        \caption{Summary statistics for immigrants in pooled Canadian census data (1901, 1911, 1921) in columns (1)-(3) and Chinese immigrants in the Chinese Register (1886-1923) in columns (4)-(5) \citep{census1901, census1911, census1921,chineseregister}.}
        \label{tab:summstats}
        % latex table generated in R 4.0.3 by xtable 1.8-4 package
% Mon Oct 10 23:48:19 2022
\begin{tabular}{lccc}
  \hline
 & CA Census (1901-1921) & Chinese Registry (1885-1924) & US Census (1900-1920) \\ 
  \hline
\% Male & 0.98 & 0.98 & 0.95 \\ 
   & (0.16) & (0.15) & (0.21) \\ 
  \% Married & 0.55 & - & 0.44 \\ 
    & (0.5) & - & (0.5) \\ 
  Avg. Age at Imm & 23.20 & 25.95 & 20.92 \\ 
     & (8.99) & (21.86) & (9.78) \\ 
  \% Can Read & 0.68 & - & 0.77 \\ 
        & (0.47) & - & (0.42) \\ 
  \% Labourers & 0.20 & 0.62 & 0.32 \\ 
           & (0.4) & (0.48) & (0.47) \\ 
   \hline
Obs &   4230 &  90946 & 186523 \\ 
   \hline
\end{tabular}

        \begin{tablenotes}
            \item *Calculated as a percentage of those over 18.
            
            \item \textbf{Notes:} Column (1) pools Canadian census data (as described in section 3.1.2) from 1901, 1911, 1921, calculating the mean of each variable for all individuals born outside of Canada, weighted based on the coverage of that year's census (5\% for 1901-1911 and 4\% for 1921). 
            Column (2) repeats this exercise for individuals born in Japan, and Column (3) repeats this exercise for individuals born in China.
            Column (4) uses the Chinese Register (as described in section 3.1.1) to calculate the mean of each variable for the population of Chinese immigrants who arrived in Canada between 1886 and 1923 and paid a non-zero Head Tax fee. Column (5) repeats this exercise for Chinese immigrants who arrived in Canada between 1886 and 1923 and paid no Head Tax fee.
        \end{tablenotes}
    \end{threeparttable}
    }
\end{table}

Table \ref{tab:summstats} summarizes various characteristics of the immigrant population in the Census and the Chinese Register. Columns (1)-(3) use the pooled 1901-1921 Censuses to compare the entire immigrant population, defined as those who were born outside of Canada, with the Japanese and Chinese immigrant populations, defined as those who were born in Japan and China respectively. Relative to all immigrants, Chinese immigrants are overwhelmingly (98\%) male, slightly less likely to be married, and less likely to have immigrated in earlier years or as children. Chinese immigrants are also significantly less likely than all immigrants to be literate (68\%), and are more likely to be working as laborers (32\%) and earning less. Japanese immigrants are more similar to Chinese immigrants than other immigrants among all dimensions.

Columns (4)-(5) use the Chinese Register data to compare Chinese immigrants who immigrated between 1886 and 1923 and were required to pay the Head Tax, to those who immigrated during the same time period but were not required to pay the Head Tax. As described in Section 2.1, exemptions from the Head Tax were granted for certain occupations such as merchants or students, which is further confirmed by the fact that while over 80\% of taxpayer Chinese immigrants were laborers, less than 5\% of non-taxpayers were laborers.\footnote{It is possible that the occupation listed in the Chinese Register referred to an immigrant's occupation prior to arrival in Canada rather than their planned occupation upon arrival in Canada, which may be why any non taxpayers at all were recorded as laborers.} The wife and children of exempted Chinese immigrants were also exempted from paying the Head Tax, which resulted in relatively more women and children among non-taxpayers relative to taxpayers. 
Overall, the characteristics of Chinese immigrants as recorded in the Chinese Register are relatively similar to those recorded in the Canadian census -- both groups are overwhelmingly male and have similar age and year of arrival distributions, although Chinese immigrants in the census are more likely to have arrived as children, indicating disproportionate outmigration of adult Chinese immigrants relative to children. The Chinese Register records significantly more laborers than the census, however this is likely due to how occupations were defined in each data set.

% Columns (3) and (4) compare the immigrant populations of the two countries. While also largely similar in terms of the gender distribution (slightly more male than the overall population), marital status (slightly more likely to be married than the overall population), there are several differences. The first and most significant is the difference in year of immigration -- immigrants to Canada immigrated nine years later on average. Additionally, while immigrants in both countries are older than the overall population, this is more pronounced in the US, and immigrants to the US are slightly less likely to be literate in English. 

% Columns (5) and (6) compare the Chinese immigrant population of the US and Canada, focusing on those who immigrated after 1885. Observe that many of the differences between these two columns could be explained by the enforcement of the Chinese Exclusion Act in the US starting in 1882: as the inflow of lower-skilled Chinese laborers stops, the fraction of laborers in the Chinese immigrant population drops, the likelihood of literacy increases, the average age increases, and the population is slightly less male. Note, however, that this explanation does not account for the differences in likelihood of being married, and furthermore, the presence of some laborers in the post-1885 US sample suggests imperfect compliance to the Chinese Exclusion Act. 

% Finally, column (7) summarizes available data from the Chinese Register. The results show that while the immigrants recorded in the Chinese Register are similarly male and arrive at similar times as the Chinese immigrants recorded in the Canadian census, they are much younger and also possibly more likely to be laborers.\footnote{For census data, I directly impute laborer status as a function of occupational group, which is encoded differently across censuses, and is therefore not directly comparable to the laborer variable in the Chinese Register, where ``Laborer'' was a category of occupation that was directly recorded. Additionally, this variable in the Chinese Register was recorded at the time of entry, and does not reflect the occupation an immigrant may have had after arrival in Canada.}
% Note that the key difference between the Canadian census and the Register is that while the Register counts each (legal) entry into the country, the census is cross-sectional and therefore does not count immigrants who both arrive and subsequently leave the country between censuses. Indeed, a simple accounting exercise shows that between 1911 and 1921, the population of Chinese immigrants represented in the census grew by nearly 10,000, while in the same time period, the Chinese Register recorded the arrival of over 29,000 Chinese immigrants, implying a net outflow of 19,000 Chinese immigrants. Repeating this same exercise implies similarly large outflows in other decades in this period. 
