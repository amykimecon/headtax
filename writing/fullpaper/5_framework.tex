\section{Selection into Immigration: Theoretical Framework}
Having established that the Head Tax reduced immigration from China, I now turn to the question of which immigrants were excluded by this increased cost. I start with the standard Roy-Borjas model of selection, which frames the decision to migrate as a tradeoff between the wage gains to migrating and the cost of migration.\footnote{I follow the notation of \citet{chiquiarhanson2005} throughout.} In particular, a prospective Chinese migrant will choose to immigrate to Canada iff

\begin{equation}
    \label{eq:model1}
    \ln (w_{Canada}) - \ln(w_{China}) - \pi > 0
\end{equation}

where $w_c$ represents the wage a worker would earn in country $c$, and $\pi$ represents the time-equivalent cost of migration (the number of hours a worker would need to work to afford to migrate), which \citet{borjas1987} models as constant. Wages are modelled as

\begin{equation}
    \label{eq:model2}
    \ln (w_c) = \mu_c + \delta_c s
\end{equation}

where $\mu_c$ represents the country-specific baseline wage in country $c$ and $\delta_c$ represents the country-specific return to skill level $s$. Substituting equation \ref{eq:model2} into equation \ref{eq:model1} yields the following condition for migration:

\begin{equation}
    \label{eq:model3}
    \mu_{Canada} + \delta_{Canada}s - \pi > \mu_{China} - \delta_{China}s
\end{equation}

\textbf{Case 1: Positive Selection into Migration} -- In the case where $\delta_{Canada} > \delta_{China}$, i.e. if the returns to skill in Canada are greater than in China, then migration from China to Canada occurs only if skill exceeds some cutoff level $\theta \equiv \frac{ -(\mu_{Canada}-\mu_{China}-\pi)}{\delta_{Canada} - \delta_{China}}$. In this case, migrants will be more skilled on average relative to the Chinese population, so a flatter wage distribution in China relative to Canada generates positive selection.

\textbf{Case 2: Negative Selection into Migration} -- Conversely, if $\delta_{China} > \delta_{Canada}$, i.e. if the returns to skill in Canada are lower than in China, then migration from China to Canada occurs only for those with $s < \theta$, generating negative selection.\footnote{Note that under both types of selection, very high values of $\pi$ or very low wage returns to migration generate the result that nobody migrates. Similarly, very low values of $\pi$ or very high wage returns to migration generate the result that everybody migrates. Because I do not observe either outcome (i.e. I do not observe either everybody migrating or nobody migrating) in any of the years in which I have data, I disregard these edge cases.}

While it is difficult to directly estimate the returns to skill in China relative to Canada in the 19th and early 20th centuries due to a dearth of available Chinese historical data, estimates from \citet{chancelpiketty2021} suggest that the two countries had similar levels of income inequality overall.\footnote{\citet{chancelpiketty2021} estimate that in 1880, the bottom 50\% of the income distribution held 17.3\% of pre-tax national income in China, and 12.6\% in Canada. However, they also estimate that the top 10\% of the income distribution in China held 50.0\% of pre-tax national income, versus only 39.0\% in Canada. On net, these benchmarks yield estimated Gini indices (measures of income inequality) of 0.49 and 0.48 for China and Canada respectively.}
In practice, however, the returns to skill for Chinese immigrants were likely lower than the returns to skill for the general Canadian population and therefore lower than the returns to skill for the general Chinese population, likely due to language barriers and overt discrimination.\footnote{For instance, age-adjusted estimates of the return to literacy using the 1901 Canadian census (the earliest year with earnings data) suggest that while for the average adult man, literacy conditional on age boosted annual earnings by \$238, for the average Chinese immigrant adult man, literacy was associated with only a \$69 increase in annual earnings.} As a result, the standard Roy-Borjas model would predict negative selection on skill for immigrants from China to Canada ($\delta_{China} > \delta_{Canada}$).

Note that in this model, the cost of migration $\pi$ and the baseline wage difference between Canada and China only affect the migration decision through the cutoff point $\theta$, which affects only the extent of selection, rather than the direction. For instance, under negative selection ($\delta_{China} > \delta_{Canada}$), $\theta$ is decreasing in $\pi$. An increase in $\pi$, such as the one generated by the implementation of the Head Tax, would decrease the skill cutoff for immigration, and selection would become even more negative, since the higher-skilled end of the migrant pool no longer experiences high enough wage gains from migration to compensate for the higher cost.


% with homogeneous migration costs generates the prediction that immigrants from countries with a higher level of income inequality than Canada will be negatively selected on skill relative to the native population, since the origin country offers a higher return to skill than Canada. Conversely, immigrants from countries with more compressed wage distributions than Canada will be positively selected on skill \citep{roy1951,borjas1987}, since they stand to gain from immigrating to a country with relatively higher returns to skill.

% Two pieces of evidence, however, complicate this homogeneous cost framework. The first is that while the returns to skill for prospective Chinese migrants may have been lower in Canada than in China, the baseline wage in Canada for the majority of workers would have been significantly higher. 

\citet{chiquiarhanson2005} extend the Roy-Borjas model to allow time-equivalent migration costs to vary by immigrant skill level:

\begin{equation}
    \ln(\pi) = \mu_{\pi} - \delta_{\pi}s
\end{equation}

In particular, the model assumes that baseline costs $\mu_{\pi}$ decrease in skill at a rate of $\delta_{\pi}$, whether due to lower wages, liquidity constraints, or ease of adaptation to the receiving country. Now migration occurs iff

\begin{equation}
    \mu_{Canada} + \delta_{Canada}s - \exp(\mu_{\pi}-\delta_{\pi}) > \mu_{China} - \delta_{China}s
\end{equation}

and under negative selection with sufficiently high migration cost, the poorest prospective migrants can no longer afford to migrate. This generates intermediate selection if $\delta_{China} > \delta{Canada}$ -- for the highest-skilled workers, the minimal wage gains from migrating to a more equal country are not enough to justify even the reduced cost of migration, while the lowest-skilled workers, who stand to gain the most from migration to a country with relatively lower returns to skill, are excluded by the high cost.

Under this framework, the Head Tax would affect selection into immigration in several ways. At one end of the initial migrant pool, the higher cost wwould unambiguously push low-skilled migrants out, making the migrant pool more positively selected relative to before the Head Tax. At the other end, the higher cost would still discourage those with less to gain from migrating, resulting in more negative selection, but the effect would be defrayed by higher skill levels. Additionally, several high-skill occupations were exempt from the Head Tax, such as merchants and students, effectively increasing $\delta_{\pi}$ and potentially inducing some higher-skilled workers to migrate. 
The magnitude of the effect will depend on each of these components individually, but unless the increase in $\delta_{\pi}$ is very small, the net effect of the Head Tax with skill-varying migration costs under initially intermediate selection would be such that there is an increase in the average skill level of the migrant pool, i.e. the tax will make selection more positive.\footnote{Observe that both the \citet{borjas1987} and \citet{chiquiarhanson2005} models predict a reduction in immigration as a result of the Head Tax, which is what I find in section 4.}

The \citet{borjas1987} and \citet{chiquiarhanson2005} models each generate distinct predictions for how an exogenous increase in migration costs would affect the selection of Chinese migrants into immigrating to Canada. The \citet{borjas1987} model predicts that the more skilled migrants would no longer choose to immigrate, leaving only the lowest-skilled migrants, leading to a decrease in the average skill of the already negatively-selected immigrant pool. The \citet{chiquiarhanson2005} model predicts that the reduction in high-skilled immigrants would be dominated by the reduction in low-skilled migrants who can no longer afford to migrate, leading to an increase in the average skill of the initially intermediately selected immigrant pool. In the following sections, I use the exogenous variation in migration costs generated by the Head Tax and its subsequent amendments to test between these two models.

