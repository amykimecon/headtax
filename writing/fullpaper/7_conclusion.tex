\section{Conclusion}

In this paper, I find evidence that the Chinese Head Tax of 1885 had significant effects on the inflow of Chinese immigrants to Canada, reducing immigration by an estimated 80\% at the peak of the Head Tax. I also find that while initially immigrants from China to Canada were neutrally or slightly positively selected on the basis of height relative to the broader Chinese population, as the Head Tax grew, Chinese immigrants became increasingly positively selected. Relative to other immigrant groups, I also find that Chinese immigrants who arrived under higher Head Taxes were less likely to work in the relatively low-paying job of laborer, and more likely to be able to afford their own house, while the effect of the Head Tax on literacy was more mixed. 

My findings indicate that migration cost plays an important role in migration decisions, and that higher migration costs in the late 19th and early 20th centuries excluded mostly immigrants with less health human capital (measured by height), lower occupational income, and less wealth. The shift towards positive selection caused by an increase in migration costs can also be seen in the composition of immigrants today, which in the U.S. and Canada has grown increasingly positively selected over time as migration costs have increased. 

While there is still a great deal of room to dig further into other effects of migration costs, such as how the selection aspect of migration costs interacts with cultural or wage assimilation after arrival, I believe that the initial findings give some insight into the direct and indirect effects of the head tax, and the possible implications for present-day immigration policy. Selection into migration is a topic of great interest, and bears significant relevance to present-day immigration policies, many of which explicitly approve immigrants based on educational background and economic standing. Understanding how immigration costs, whether explicit (as in the case of the head tax), or implicit (as in the case of modern-day immigration), play a role in selection into migration and outcomes after arrival, is key to understanding how to best shape equitable immigration policy going forward. 
