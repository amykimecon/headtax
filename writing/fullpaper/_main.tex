\documentclass[12pt]{article}
\usepackage{../preamble}

% % where graphs are stored
% \graphicspath{{20230410/}}

\title{Who Pays the Cost of Exclusion?: Selection into Immigration Under the 1885 Chinese Head Tax}
\author{Amy Kim}
\date{\today}

\begin{document}

%%% TITLE
\maketitle

%%% ABSTRACT
\begin{abstract} \footnotesize
    As immigration pathways to the U.S. and Canada become increasingly complex, costs, ranging from administrative fees to skill restrictions, play an increasingly central role in immigration policy. Despite their importance, however, the complexity and heterogeneity of modern-day immigration systems have made it difficult to assess whether these costs reduce migration, and if so, how the composition of the immigrant pool is affected. 
    In this paper, I study a historical Canadian immigration policy, the 1885 Chinese Head Tax, which imposed a well-documented and time-varying per-person fee on Chinese immigrants to Canada, to estimate how increases in migration costs affected Chinese immigration to Canada.
    I begin by estimating the impact of the Head Tax on the inflow of Chinese immigrants to Canada and find that at its peak, the Head Tax reduced Chinese immigration to Canada by nearly 80\%. I then document changes in the composition of the Chinese immigrant pool in Canada. I find that as the cost of migration increased, Chinese immigrants to Canada became more positively selected on the basis of height (a common measure of human capital in economic history) as well as occupation, literacy, and home ownership. 
    This paper builds on the literature studying selection into migration by using precise changes in cost and detailed microdata on immigrants at the time of arrival, rather than relying on proxies for cost or cross-sectional census data alone. I also build on the historical immigration literature by documenting patterns of Chinese immigrants in particular -- an important but understudied group central to the evolution of exclusionary immigration policy in the U.S. and Canada.
\end{abstract}

% \newpage
%%% CONTENTS
\import{./}{1_intro.tex}
\import{./}{2_context.tex}
\import{./}{3_data.tex}
\import{./}{4_flows.tex}
\import{./}{5_framework.tex}
\import{./}{6_selection.tex}
\import{./}{7_conclusion.tex}

%%% BIBLIOGRAPHY
\newpage
\bibliography{../headtax}

%%% APPENDIX
\newpage
\appendix 
\import{./}{a1_appfigs.tex}

\end{document}