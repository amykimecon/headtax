\documentclass[12pt]{article}
\ProvidesPackage{preamble}

%% General formatting
\usepackage{enumerate}
\usepackage{amsmath}
\usepackage{breqn}
\usepackage{bbm}
\usepackage[margin=1in]{geometry}
\usepackage[hidelinks]{hyperref}
\usepackage{subfiles}
\usepackage{setspace}
\usepackage{graphicx}
\usepackage[labelfont=bf]{caption}
\usepackage{hhline}
\usepackage{threeparttable}
\usepackage{tabularx}
\usepackage{parskip}

% %% Beamer Stuff
% \usepackage{appendixnumberbeamer}å

%% Bibliography packages
\usepackage{natbib}
\bibliographystyle{abbrvnat}
\setlength\parindent{0pt}
\onehalfspacing

\title{ECO 523 Short Paper: The 1885 Chinese Head Tax and Immigration to Canada}
\author{Amy Kim}

\begin{document}
\maketitle

\section{Introduction}

[put lit review here]

\section{Background and Historical Context}
Although the history of Chinese immigration to North America reaches back to the 1700s, it was not until the Gold Rush of the 1850s that large numbers of Chinese immigrants began to settle along the West Coast. 
While initially concentrated in California, later discoveries of gold in British Columbia expanded Chinese immigration to Canada, and by 1860 there were an estimated 7,000 Chinese inhabitants of British Columbia \citep{chan2019}.
Following the Gold Rush, Chinese immigrants largely took on low-wage jobs in the manufacturing and service industries, and in 1880-1885 contributed heavily to the construction of the Canadian Pacific Railway.
As the completion of the railway approached, however, sinophobic sentiment began to foment, and in 1884 a commission was formed to investigate the possibility of restricting Chinese immigration to Canada \citep{chan2016}.

\subsection{Chinese Head Tax in Canada}
With the passage of the Chinese Immigration Act of 1885 came a \$50 `Head Tax' -- a per-person entry fee for all immigrants from China. While the commission appointed to investigate Chinese immigration had originally suggested a \$10 head tax, intended to pay for a health inspector at entry ports,
it was clear that the \$50 fee was intended to dissuade Chinese immigrants from settling in Canada. 50 CAD in 1885 would be worth approximately 1,500 USD in 2023, and at the time was nearly a fifth of the average Chinese immigrant's yearly salary.
As Chinese immigration to Canada continued to grow despite the tax, the government raised the tax to \$100 in 1900 (approximately 3,500 USD in 2023) and to \$500 in 1903 (approximately 17,000 USD in 2023). \\ 

The Chinese Immigration Act also limited the capacity of incoming ships carrying Chinese passengers (with no changes to capacity for incoming ships carrying European passengers) and while exceptions to the tax were made for diplomats, merchants, students, and others, the vast majority of Chinese immigrants were forced to pay the tax. 
Failure to do so would result in being sent back to China, and as a result many immigrants had to either borrow money or enter indentured servitude contracts to cover the cost of the tax. 
The tax remained in place until 1923, when the Chinese Immigration Act of 1923 banned Chinese immigration to Canada altogether until its repeal in 1947.

\subsection{Chinese Immigration to the US}
Chinese immigration to the US evolved in a similar way as Canada throughout the 19th century, beginning with the Gold Rush, growing through the construction of the Transcontinental Railroad, and culminating in the Chinese Exclusion Act of 1882.
The legislation in the US, however, took a more severe approach than the Canadian counterpart, completely banning the immigration of Chinese laborers, a ban which remained in place until 1943. 
While there are some accounts of immigrants from China landing in Canada and crossing the border by land into the US, estimates suggest that the Exclusion Act effectively stopped the growth of the Chinese population in the US.

\section{Data}
\subsection{Description of Data Sources}

\subsubsection{Chinese Register}
The first and primary source of data that I use is the Register of Chinese Immigrants to Canada\footnote{Data is obtained from \citet{chineseregister}.}, 
which is a record maintained from 1885 to 1949 of all Chinese immigrants to Canada.
The Register includes the full name, amount of tax paid, age and sex, occupation, place of origin within China, port and date of arrival, among other variables. 
In addition to serving as a full count of immigrants of Chinese origin who \textit{entered} Canada during the years affected by the Head Tax,
the Register also records any Chinese immigrants who had already entered Canada prior to the implementation of the Chinese Immigration Act in 1885 (and therefore were not subject to the tax),
but chose to pay the tax and register with immigration authorities could exit and re-enter the country without paying additional fees.
Although this dataset allows us to precisely observing the timing of entry into Canada, it does not record the outflow of Chinese immigrants, and so it does not accurately capture the stock of Chinese immigrants in Canada in any given year.

\subsubsection{Canadian Census}
Microdata from historical Canadian censuses are currently available through 1921.\footnote{Data are compiled from \citet{census1881}, \citet{census1891}, \citet{census1901}, \citet{census1911}, \citet{census1921}, for the 1881, 1891, 1901, 1911, and 1921 censuses respectively.} 
Although all census data from 1881 has been fully digitized, resulting in 100\% coverage of the population at the time, only 4-5\% of subsequent censuses have been digitized.
While this data source provides a rich set of variables, the low sampling rate results in a very small sample size for Chinese immigrants, who were already a relatively small percentage of the Canadian population at the time.
Since the census only took place every ten years, this data source provides a cross-sectional view of the country at different points in time, acting as a complement to the data on inflows from the Chinese Register.
Section 3.2 compares these two datasets in further detail.

\subsubsection{US Census}
Finally, I also use US Census data in order to compare the characteristics of Chinese immigrants to Canada to the characteristics of Chinese immigrants to another country.
I use full-count census data from the 1880-1920 US censuses,\footnote{All data is obtained from \citet{ipums}.} 

\subsection{Comparing Data Sources}

\begin{table}[!h]
    \centering 
    \caption{Summary statistics for }
    \resizebox{\textwidth}{!}{
    % latex table generated in R 4.0.3 by xtable 1.8-4 package
% Mon Oct 10 23:48:19 2022
\begin{tabular}{lccc}
  \hline
 & CA Census (1901-1921) & Chinese Registry (1885-1924) & US Census (1900-1920) \\ 
  \hline
\% Male & 0.98 & 0.98 & 0.95 \\ 
   & (0.16) & (0.15) & (0.21) \\ 
  \% Married & 0.55 & - & 0.44 \\ 
    & (0.5) & - & (0.5) \\ 
  Avg. Age at Imm & 23.20 & 25.95 & 20.92 \\ 
     & (8.99) & (21.86) & (9.78) \\ 
  \% Can Read & 0.68 & - & 0.77 \\ 
        & (0.47) & - & (0.42) \\ 
  \% Labourers & 0.20 & 0.62 & 0.32 \\ 
           & (0.4) & (0.48) & (0.47) \\ 
   \hline
Obs &   4230 &  90946 & 186523 \\ 
   \hline
\end{tabular}

    }
\end{table}


\section{Descriptive Analysis}


\section{Comparing Canada to the US}

\section{Conclusion}

\newpage

\bibliography{../headtax.bib}

\newpage 
\appendix
\section{Details on Canadian Census Harmonization}
\begin{table}[!h]
    \centering 
    \begin{tabular}{lccccc}
    \hhline{======} 
                & \multicolumn{5}{c}{Census Year (Coverage)} \\
                & 1881 (100\%) & 1891 (5\%) & 1901 (5\%) & 1911 (5\%) & 1921 (4\%) \\
    \hline 
    \end{tabular}    
\end{table}


\end{document}
